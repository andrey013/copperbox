
\documentclass{article}

\usepackage{amssymb}

\begin{document}
\title{Mullein User Guide}
\maketitle

\section{Introduction}
\subsection{What is Mullein?}
Mullein is set of modules for music description language written 
in Haskell. Mullein is primarily inspired by Paul Hudak's 
Haskore system, but Mullein has a different objective to Haskore: 
Mullein describes musical structures for rendering as printed 
scores; Haskore describes musical structures for 
\emph{performance}, where a perfomance is typically rendered to 
a sound file. 

Ideally Mullein would not be a separate system to Haskore, but
an extension allowing Haskore to generate scores as well 
as performances. Unfortunately Haskore with its emphasis on 
performance is not well suited to describing scores. Valuable
musical information for score printing is not readily available
in Haskore. The most obvious example is duration - Haskore 
records note durations as absolute values, in printed scores 
durations are symbolic (a eighth note always has the same 
symbolic duration \(\frac{1}{8}\) regardless of the tempo 
of the music). While is easy to recover note durations from 
simple Haskore performances it is very difficult to recover 
durations in if a score has trills or other ornaments.

An early prototype of Mullein tried to alleviate this problem 
by allowing Haskore performances to be augmented with rewriting 
rules to recover symbolic durations from absolute ones. But the 
additional machinary was excessive and hardly intuitive, 
strongly suggesting that abandoning compatibility with Haskore
was best option for Mullein.

\section{section b}

unit note length

pitch relative to scale 

\section{section c}
Duration - where consecutive notes share the same duration, only
the first duration needs to be specified. 


\end{document}
