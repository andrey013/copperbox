
\documentclass{article}

\usepackage{amssymb}
\usepackage{nicefrac}
\usepackage{alltt}
\usepackage{comment}


\begin{document}
\author{Stephen Peter Tetley}
\title{Mullein User Guide}
\maketitle
\tableofcontents

\section{Introduction}
\subsection{What is Mullein?}
Mullein is set of modules for music description language written 
in Haskell. Mullein is primarily inspired by Paul Hudak's 
Haskore system, but Mullein has a different objective to Haskore: 
Mullein describes musical structures for rendering as printed 
scores; Haskore describes musical structures for 
\emph{performance}, where a perfomance is typically rendered to 
a sound file. 

Ideally Mullein would not be a separate system to Haskore, but
an extension allowing Haskore to generate scores as well 
as performances. Unfortunately Haskore with its emphasis on 
performance is not well suited to describing scores. Valuable
musical information for score printing is not readily available
in Haskore. The most obvious example is duration - Haskore 
records note durations as absolute values, in printed scores 
durations are symbolic (a eighth note always has the same 
symbolic duration \nicefrac{1}{8} regardless of the tempo 
of the music). While is easy to recover note durations from 
simple Haskore performances it is very difficult to recover 
durations in if a score has trills or other ornaments.

An early prototype of Mullein tried to alleviate this problem 
by allowing Haskore performances to be augmented with rewriting 
rules to recover symbolic durations from absolute ones. But the 
additional machinary was excessive and hardly intuitive, 
strongly suggesting that abandoning compatibility with Haskore
was best option for Mullein.

\begin{comment}

unit note length

pitch relative to scale 

Mullein represents durations symbolically... 
\begin{alltt} 
longa, breve, \nicefrac{1}{1}, \nicefrac{1}{2}, \nicefrac{1}{4},
\nicefrac{1}{8}, \nicefrac{1}{16}
\end{alltt}
\end{comment}




\section{Micro guide to LilyPond}
\subsection{Pitch}
LilyPond has two modes for representing octaves - relative and 
absolute - with relative mode appearing to be most popular for 
score authors. Relative mode allows slightly more concise scores, 
as octave annotations are only needed when the pitch change 
between two successive notes is greater than five half-steps.

\section{Micro guide to ABC}
\subsection{Pitch}
Middle C is represented by the symbol \verb+C+.
The two octaves spanning the treble clef are represented
with the pitches \verb+C-Bc-b+. Pitches outside this range are 
formed by appending commas or apostrophes. Each comma appended 
lowers the octave, e.g. \verb+C,,+ is c two octaves below middle 
C. Similarly appending apostrophes raises the octave \verb+c'+ 
is one octave above \verb+c+ (and two octaves above middle C). 
ABC allows equivalent octave spellings, although the merit of 
this in practice is questionable. These alternatives all denote 
middle C: \verb+C  c,  C,'  c,,'  c,',+

ABC tunes have a key signature defined in the header prologue 
(if the key signature is not specified it defaults to C major).
With this context established, notes only need to be augmented 
with accidentals if the accidental needs printing. For instance, 
in G major the note \verb+F+ in the score represents the 
absolute pitch F$\sharp$, but following notational conventions 
the sharp is ellided, a musician knows to play it as 
F$\sharp$ from the key signature. In G major, \verb+=F+ prints 
F$\natural$ and \verb+^F+ prints F$\sharp$, decorating the F 
with a cautionary (though strictly superfluous) sharp. 

\subsection{Duration}
Durations in ABC tunes are denoted by multiplers of the 
\emph{unit note length} rather than absolute values. This is a 
shorthand to reduce typing, no annotation is needed to notes 
whose duration equals the unit note length. A note G with 
duration double the unit note length is \verb+G2+, with duration 
half the unit note length it is \verb+G/2+.

The unit note length is either specified in the header of the 
tune, or derived from the meter. The derivation is: 
if the meter expressed as a decimal is $\geqq$ 0.75 then the 
unit note length is an eighth, otherwise it is a sixteenth.

\subsection{Metrical division: Beaming and Bar lines}
Notes to be joined within a beam group must be grouped together
without spaces, which makes the ABC format whitespace sensitive!
ABC is permissive in its interpretation of a beam group: if the 
group includes a note that cannot be beamed (say a quarter or half
note) it will be printed singularly and the beam group will be 
resumed. The flipside of this flexibility with beam groups is a
score is free to designate beam groups that have no correlation  
with the pulsation implied by a tune's meter.

Barring is entirely at the mercy of the user. Bars are be printed 
where they are placed, neither \verb+abc2ps+ or \verb+abcm2ps+ 
appear to give warnings when bars are too long or too short for 
the meter.


\end{document}
