\documentclass{article}

\usepackage{comment}
\usepackage{amssymb}


\begin{document}

\title{Wumpus-Core Manual}
\maketitle

%-----------------------------------------------------------------




%-----------------------------------------------------------------
\section{Wumpus-Core graphics model}
%-----------------------------------------------------------------

Wumpus-core uses a small set objects to build \emph{Pictures} - 
2D vector graphics renderable to PostScript or SVG (Scalable 
Vector Graphics). 

The most primitive objects are \emph{points} and \emph{vectors} 
to displace them. Points are used to describe \emph{paths} which 
are made from stright line segments or cubic Bezier curves. Paths 
can be \emph{stroked} (drawing the outline), \emph{filled} 
(colouring in the shape bounded by the path) or \emph{clipped} 
which turns the path into a mask to crop pictures with. 
\emph{Labels} contain printed text. Labels and stoked, filled 
or clipped paths are considered to be \emph{primitives}, which 
are aggregated to make \emph{pictures}. Pictures themselves can 
be aggregated to making composite pictures (Wumpus-core represents 
pictures as a tree, branches can contain one or more pictures, 
leaves contain primitives).
 


%-----------------------------------------------------------------
\section{PostScript SVG}
%-----------------------------------------------------------------

\begin{tabular}{l l}
[ $e_{0}X$ $e_{0}Y$ $e_{1}X$ $e_{1}Y$ $oX$ $oY$ ] concat & 
matrix($e_{0}X$, $e_{0}Y$, $e_{1}X$, $e_{1}Y$, $oX$, $oY$) \\
$x$ $y$ moveto & M $x$ $y$ \\
$x$ $y$ lineto & L $x$ $y$ \\
$x_{1}$ $y_{1}$ $x_{2}$ $y_{2}$ $x_{3}$ $y_{3}$ curveto  & 
C $x_{1}$ $y_{1}$ $x_{2}$ $y_{2}$ $x_{3}$ $y_{3}$ \\
\end{tabular}




%-----------------------------------------------------------------
\section{References}
%-----------------------------------------------------------------

PostScript is a registered trademark of Adobe Systems Inc.

\end{document}
